\documentclass[12pt]{article} 

\usepackage{amsmath,amsfonts,amssymb} 
\usepackage{amsthm}
\usepackage{enumerate}

\theoremstyle{definition}
\newtheorem{definition}{Definition}[section]
\newtheorem{proposition}[definition]{Proposition}

\begin{document}

\title{Math 220 Proof Portfolio}
\date{}

\maketitle

%%%%%
%%%%% Entry 1
%%%%%

\section{The first entry (change this section name)}

Put your first entry here, unless you are choosing not to revise it from the version submitted with Problem Set $6$. If you are not revising Entry $1$, you can leave this section blank (or you can paste in your Entry $1$ and write at the top that you have not revised it).

You should change all uses of \texttt{$\backslash$section} to \texttt{$\backslash$subsection} in your \LaTeX\ source, since the entire entry is now one section of the overall portfolio. As an example, I've pasted the first chunk of the template source code below.

\subsection*{Introduction}

The purpose of this document is to establish the irrationality of a certian class of real numbers, namely roots of prime numbers. First recall some basic facts and definitions.

\begin{enumerate}
\item
A real number $a$ is called \textit{rational} if there exist integers $m,n$ with $n \neq 0$ and  $ a = \frac{m}{n}$.
\item
(\textit{B\'ezout's identity})
If $a$ and $b$ are two integers, not both $0$, then there exist integers $u,v$ such that
$$a u + bv = \gcd(a,b).$$
Furthermore, if there exist $u,v \in \mathbb{Z}$ such that $au+bv=1$, then $\gcd(a,b) = 1$.
\item
(\textit{Euclid's lemma}, weak form)
If $p$ is a prime number and $a,b$ are integers such that $p \mid ab$, then either $p \mid a$ or $p \mid b$.
\end{enumerate}

\subsection{Preliminary results}

We now prove two preliminary lemmas.\\

[etc.]


%%%%%
%%%%% Entry 2
%%%%%
\newpage
\section{The second entry (change this section name)}

Write your second entry here, unless you submitted a draft at the end of classes and do not wish to revise it.

%%%%%
%%%%% Entry 3
%%%%%
\newpage
\section{The third entry (change this section name)}

Write your third entry here.

%%%%%
%%%%% Entry 4
%%%%%
\newpage
\section{The fourth entry (change this section name)}

Write your fourth entry here.


%%%%%
%%%%% Reflection
%%%%%
\newpage
\section{Reflection essay}

Write your short essay refleting on mathematical reasoning and proof here.


\end{document}
